\documentclass[fontsize=12pt, paper=a4, headinclude, twoside=false, parskip=half+, pagesize=auto, numbers=noenddot, open=right, toc=listof, toc=bibliography]{scrreprt}
% PDF-Kompression
\pdfminorversion=5
\pdfobjcompresslevel=1
% Allgemeines
\usepackage[automark]{scrpage2} % Kopf- und Fußzeilen
\usepackage{amsmath,marvosym} % Mathesachen
\usepackage[T1]{fontenc} % Ligaturen, richtige Umlaute im PDF
\usepackage[utf8]{inputenc}% UTF8-Kodierung für Umlaute usw
% Schriften
\usepackage{mathpazo} % Palatino für Mathemodus
%\usepackage{mathpazo,tgpagella} % auch sehr schöne Schriften
\usepackage{setspace} % Zeilenabstand
\onehalfspacing % 1,5 Zeilen
% Schriften-Größen
\setkomafont{chapter}{\Huge\rmfamily} % Überschrift der Ebene
\setkomafont{section}{\Large\rmfamily}
\setkomafont{subsection}{\large\rmfamily}
\setkomafont{subsubsection}{\large\rmfamily}
\setkomafont{chapterentry}{\large\rmfamily} % Überschrift der Ebene in Inhaltsverzeichnis
\setkomafont{descriptionlabel}{\bfseries\rmfamily} % für description-Umgebungen
\setkomafont{captionlabel}{\small\bfseries}
\setkomafont{caption}{\small}
% Sprache: Deutsch
\usepackage[ngerman]{babel} % Silbentrennung
% PDF
\usepackage[ngerman,pdfauthor={Nils Gawlik},  pdfauthor={Nils Gawlik}, pdftitle={TODO}, breaklinks=true,baseurl={}]{hyperref}
\hypersetup{
    colorlinks=true,
    linkcolor=blue,
    filecolor=magenta,      
    urlcolor=cyan,
}
\urlstyle{same}

\usepackage[final]{microtype} % mikrotypographische Optimierungen
\usepackage{url} % ermögliche Links (URLs)
\usepackage{pdflscape} % einzelne Seiten drehen können
% Tabellen
\usepackage{multirow} % Tabellen-Zellen über mehrere Zeilen
\usepackage{multicol} % mehrere Spalten auf eine Seite
\usepackage{tabularx} % für Tabellen mit vorgegeben Größen
\usepackage{longtable} % Tabellen über mehrere Seiten
\usepackage{array}
%  Bibliographie
\usepackage{bibgerm} % Umlaute in BibTeX
% Bilder
\usepackage{graphicx} % Bilder
\usepackage{color} % Farben
\graphicspath{{images/}} % Lege den Standardpfad mit Bilder fest
\DeclareGraphicsExtensions{.pdf,.png,.jpg} % bevorzuge pdf-Dateien vor den anderen
\usepackage{subcaption}  % mehrere Abbildungen nebeneinander/übereinander
\usepackage[all]{hypcap} % Beim Klicken auf Links zum Bild und nicht zu Caption gehen
% Bildunterschrift
\setcapindent{0em} % kein Einrücken der Caption von Figures und Tabellen
\setcapwidth{0.9\textwidth} % Breite der Caption nur 90% der Textbreite, damit sie sich vom restlichen Text abhebt
\setlength{\abovecaptionskip}{0.2cm} % Abstand der zwischen Bild- und Bildunterschrift
% Quellcode
% für Formatierung in Quelltexten, hier im Anhang
\usepackage{listings}
\definecolor{grau}{gray}{0.25}
\lstset{
	extendedchars=true,
	basicstyle=\tiny\ttfamily,
	%basicstyle=\footnotesize\ttfamily,
	tabsize=2,
	keywordstyle=\textbf,
	commentstyle=\color{grau},
	stringstyle=\textit,
	numbers=left,
	numberstyle=\tiny,
	% für schönen Zeilenumbruch
	breakautoindent  = true,
	breakindent      = 2em,
	breaklines       = true,
	postbreak        = ,
	prebreak         = \raisebox{-.8ex}[0ex][0ex]{\Righttorque},
}
% linksbündige Fußboten
\deffootnote{1.5em}{1em}{\makebox[1.5em][l]{\thefootnotemark}}

\typearea{14} % typearea berechnet einen sinnvollen Satzspiegel (das heißt die Seitenränder usw.) siehe auch http://www.ctan.org/pkg/typearea. Diese Berechnung befindet sich am Schluss, damit die Einstellungen von oben berücksichtigt werden

\usepackage{scrhack} % Vermeidung einer Warnung


% Eigene Befehle %%%%%%%%%%%%%%%%%%%%%%%%%%%%%%%%%%%%%%%%%%%%%%%%%5
% Matrix
\newcommand{\mat}[1]{
      {\textbf{#1}}
}
\newcommand{\todo}[1]{
      {\colorbox{red}{ TODO: #1 }}
}
\newcommand{\todotext}[1]{
      {\color{red} TODO: #1} \normalfont
}
\newcommand{\info}[1]{
      {\colorbox{blue}{ (INFO: #1)}}
}
% Hinweis auf Programme in Datei
\newcommand{\datei}[1]{
      {\ttfamily{#1}}
}
\newcommand{\code}[1]{
      {\ttfamily{#1}}
}
% bild mit defnierter Breite einfügen
\newcommand{\bild}[4]{
  \begin{figure}[!hbt]
    \centering
      \vspace{1ex}
      \includegraphics[width=#2]{images/#1}
      \caption[#4]{\label{img.#1} #3}
    \vspace{1ex}
  \end{figure}
}
% bild mit eigener Breite
\newcommand{\bilda}[3]{
  \begin{figure}[!hbt]
    \centering
      \vspace{1ex}
      \includegraphics{images/#1}
      \caption[#3]{\label{img.#1} #2}
      \vspace{1ex}
  \end{figure}
}

% Bild todo
\newcommand{\bildt}[2]{
  \begin{figure}[!hbt]
    \begin{center}
      \vspace{2ex}
	      \includegraphics[width=6cm]{images/todobild}
      %\caption{\label{#1} \color{red}{ TODO: #2}}
      \caption{\label{#1} \todotext{#2}}
      %{\caption{\label{#1} {\todo{#2}}}}
      \vspace{2ex}
    \end{center}
  \end{figure}
}
 % Importiere die Einstellungen aus der Präambel
% hier beginnt der eigentliche Inhalt
\begin{document}
\pagenumbering{Roman} % Seitennummerierung mit großen römischen Zahlen 
\pagestyle{empty} % keine Kopf- oder Fußzeilen auf den ersten Seiten

% Titelseite
\clearscrheadings\clearscrplain
\begin{center}
\begin{Huge}
Entwicklung eines dreidimensionalen zyklischen Dungeon-Generators mit Unity3D\\
\end{Huge}
\vspace{8mm}
Bachelorarbeit\\
\vspace{0.4cm}
\vspace{2 cm}
Nils Gawlik \\
Matrikel-Nummer 553449\\
\vspace{8cm}
\begin{tabular}{rl}
{\bfseries Betreuer} & Prof. Lenz \\
{\bfseries Erstprüfer}& Prof. Lenz\\
{\bfseries Zweitprüfer}& Prof. Strippgen\\
\end{tabular}

\end{center}
\clearpage


\pagestyle{useheadings} % normale Kopf- und Fußzeilen für den Rest

\tableofcontents % erstelle hier das Inhaltsverzeichnis
\listoffigures % erstelle hier das Abbildungsverzeichnis
\clearpage

% richtiger Inhalt
\pagenumbering{arabic} % ab jetzt die normale arabische Nummerierung
\chapter{Einführung und Definitionen}

\section{Motivation}

In einem Blog Post von 2016 beschreibt Joris Dormans einen Ansatz zur prozeduralen Level-Generierung, den er „Cyclic Dungeon Generation“ nennt. Dieses Verfahren präsentiert er als Gegensatz zu den Verzweigungs-Ansätzen die man in vielen anderen Dungeon-Generatoren findet \cite{blogCyclic}.

Zyklische Ansätze haben viele praktische Vorteile gegenüber Verzweigungs-Ansätzen. Zum einen vermeiden sie Sackgassen und damit verbundenes Backtracking und zum anderen ermöglichen sie das Generieren von typisch zyklischen Strukturen, unter anderem: „Einbahnstraßen“ mit anderem Rückweg, alternative Wege, Abkürzungen und direkte Rückkehr zum Startpunkt des Dungeons 
\cite{blogCyclic}.

Dormans Spiel „Unexplored“, dass im Blog Post als Beispiel verwendet wird, ist ein zweidimensionales Spiel in einer Top-Down-Perspektive. Besonders für zyklische Generierung ist es aber auch interessant einen dreidimensionalen Raum zu nutzen, da dies interessante Strukturen ermöglicht wie Brücken die bereits besuchte Teile des Dungeons überspannen.  

Das Ziel dieser Bachelorarbeit ist es ein Verfahren zur zyklischen Generierung von Dungeons zu konzipieren und zu implementieren. Dieses Verfahren sollte im dreidimensionalen Raum funktionieren und die dritte Dimension für interessante Strukturen nutzen können. Die Implementierung findet in der Game Engine Unity3D statt.

\todo{Quelle finden, dass viele designte Dungeons Zyklen aufweisen}

\section{Dungeons}
\subsection{Definition von Dungeon}

Der englische Begriff „Dungeon“ wurde für diese Arbeit bewusst gewählt, er hat im Bereich Spiele eine Bedeutung, die sich von der wörtlichen Übersetzung „Verlies, Kerker“ unterscheidet. Eine einheitliche Definition für Dungeon in diesem Kontext existiert nicht, aber üblicherweise bezeichnet er einen weitläufigen, in sich geschlossenen Raum gefüllt mit Fallen und Monstern, den ein oder mehrere Spieler navigieren müssen, häufig mit einer großen Herausforderung am Ende, z.B. einem besonders starken Boss-Monster.

\bild{Warreks_Nest}{8cm}{Beispiel eines Dungeons \citeimg{warreksNestMap}}{Dungeon-Beispiel}

\subsection{Wichtige Beispiele von Dungeons}

Dungeons ist in einigen der populärsten Spielen vertreten. Beispiele sind „Dungeons \& Dragons“, „The Legend Of Zelda“, „The Elder Scrolls“ und „Minecraft“.

\todo{Eine gute Auswahl finden, Bilder-Collage machen und einfügen}

\subsection{Prozedurale Dungeon-Generierung in Spielen, Geschichte}

In Computerspielen gibt es eine relativ lange Tradition der Dungeon-Generierung.

\todo{Research Rougelikes, Roguelites, TES Oblivion}

\section{Genaue Beschreibung der Fragestellung}

Der Inhalt der Arbeit ist die Entwicklung eines dreidimensionalen zyklischen Dungeon-Generators mit Unity3D. 

Im Folgenden soll erläutert werden, wie die Einschränkungen „dreidimensional“ und „zyklisch“ im Kontext dieser Arbeit 
zu verstehen sind.

\subsection{Definition von zyklisch}\label{c.weitereskapitel}

Der Begriff zyklisch ist als deutsches Äquivalent des englischen Begriffs „cyclic“ zu verstehen und ist aus dem bereits erwähnten Blog Post von Joris Dormans \cite{blogCyclic} übernommen. 

Für diese Arbeit wird folgende Definition verwendet: Ein Dungeon-Generator is zyklisch wenn A) der generierte Dungeon Zyklen enthält und B) Zyklen als Bausteine während der Generierung dienen.

\bild{Warrek_Graph}{8cm}{Bild eines Dungeons, mit möglichem Level-Graph, der Graph ist ungewichtet und ungerichtet (\citeimg{warreksNestMap}, bearbeitet)}{Dungeon-Beispiel mit Graph}

Zyklen im Dungeon erkennt man am Level-Graph, ist der Level-Graph zyklisch, so ist auch der Dungeon zyklisch. Betrachtet man das Beispiel in Abbildung \ref{img.Warrek_Graph} kann man mehrere Zyklen erkennen. Praktisch gesehen bedeutet das, dass der Spieler im Kreis laufen kann. Die Übersetzung der Level-Geometrie in einen Graphen ist nicht eindeutig - es sind mehrere Graphen möglich, aber der Graph sollte die Topologie des Raumes wiedergeben. \todo{Evtl. etwas zu topologischer Graphentheorie schreiben}

In der Realität kommt es selten vor, dass man beurteilen muss ob ein bereits existierender Dungeon zyklisch ist oder nicht. Häufiger ergibt sich ein Level-Graph während der Generierung, z.B. indem kontinuierlich Räume hinzugefügt werden. Die Räume können als Knoten, die Verbindungen zwischen den Räumen als Kanten verstanden werden.

Der Bedingung, dass Zyklen während der Generierung als Bausteine des Dungeons dienen, ist wichtig. So wird von einer anderen Art der Generierung unterschieden, wie man sie auch häufig in Level-Generatoren findet: Man generiert ein Labyrinth ohne Zyklen mit einem Verzweigungs-Ansatz und fügt anschließend willkürlich Verbindungen zwischen Räumen hinzu, bzw. - etwas bildlicher - „reißt Wände ein“. So entstehen auch Zyklen, diese sind aber willkürlich und schlecht kontrollierbar. 

Sind allerdings die Zyklen ein Baustein des Dungeons, so dass mehrere vordefinierte Arten von Zyklen durch Aneinanderreihung oder Verschachtelung kombiniert werden, hat man Kontrolle über die Natur der Zyklen. Variablen wie Größe, Richtung und Anordnung der Zyklen sind direkt konfigurierbar.

\subsection{Dreidimensionalität}

Dreidimensionalität \todo{Wie definieren???}

\section{Warum Unity?}

Unity ist eine Game Engine für die Entwicklung von 2D und 3D-Spielen. 

Es wurde eine Game Engine verwendet, um den Implementierungs-Teil der Arbeit auf den Generator selbst und das User-Interface des Generators zu beschränken. Die Implementierung von Engine-typischen Features wie Rendering oder Kollision soll nicht Teil der Arbeit sein.

Die Wahl einer Game Engine für ein bestimmtes Projekt ist immer auch eine subjektive Entscheidung. Hauptargumente für Unity sind die weite Verbreitung der Game Engine unter Indie-Entwicklern (die häufig von prozeduraler Generierung Gebrauch machen \todo{Really?}), das Entity-Component-System, dass einen modularen Aufbau ermöglicht und das Editor Scripting, dass es erlaubt ein User Interface für den Generator innerhalb der Engine zu schreiben. 

\chapter{Der Ersetzungs-Algorithmus}

Im Folgenden wird der der Arbeit zugrunde liegende Algorithmus theoretisch erklärt.

\section{Formale Grammatiken und Graphgrammatiken für die Generierung}

Dem Algorithmus liegt ein Ersetzungssystem zu Grunde. Dieses ähnelt der Generierung durch eine Graphgrammatik, weswegen kurz auf die Prozedurale Generierung mit Hilfe von Formalen Grammatiken und im speziellen Graphgrammatiken eingegangen wird.

Formale Grammatiken werden in der prozeduralen Generierung gerne verwendet um Wörter und Texte zu generieren oder andere eindimensionale Ketten von Symbolen, wie z.B. Melodien. „Procedural Content Generation In Games“ sagt folgendes:

„Generative grammars were originally developed to formally describe structures in natural language. These structures—phrases, sentences, etc.—are modelled by a finite set of recursive rules that describe how larger-scale structures are built from smaller-scale ones, grounding out in individual words as the terminal symbols.“
\cite[Kap.~3.5, S.~45]{shaker2016procedural}

Damit eine nicht-deterministische Grammatik zur Generierung verwendet werden kann muss eine Entscheidung getroffen werden, in welcher Reihenfolge Regeln angewendet werden. Eine mögliche Methode ist jeden Generations-Schritt eine zufällige Regel aus der Menge an anwendbaren Regeln auszuwählen. 
\cite[Kap.~5.2, S.~75]{shaker2016procedural}
Eine Terminierung des Algorithmus ist hier nicht garantiert, da Regeln rekursiv sein können. Aber bei einer zufälligen Auswahl ist eine Terminierung in einer akzeptablen Laufzeit sehr wahrscheinlich, angenommen die Formale Sprache enthält keine rekursiven Regeln ohne Abbruchbedingung. 

Eine Graphgrammatik verhält sich wie eine Formale Grammatik, allerdings sind die linken und rechten Seiten hier Mustergraphen. Hierbei ist es wichtig, dass die individuellen Knoten auf der linken Seite jeweils individuelle Knoten auf der rechten Seite zugewiesen werden können, damit die Ersetzung stattfinden kann. 
\cite[Kap.~5.5.1, S.~80]{shaker2016procedural}
Durch widerholte Ersetzung von Teilgraphen kann hier aus einem Start-Graph ein End-Graph erzeugt werden. Die Menge der produzierbaren Graphen bildet die durch die Graphgrammatik definierte Formale Sprache. 

In \cite{dormansAdventures} benutzt Dormans Graphgrammatiken um Missionsstrukturen zu generieren und Figur-Grammatiken (engl. Shape Grammars) um den zugehörigen Raum zu generieren. Diese Missionsstrukturen sind zyklisch, der Raum allerdings nicht.

\section{Spezieller Algorithmus}\label{s.speziellerAlgorithmus}

Der Generator den ich diese Bachelorarbeit entwickelt habe, ist durch wiederholte Iteration und Experimentation entstanden. Diese Implementierung wird später ausführlich in Kapitel \ref{c.implementierung} beschrieben.

Aber zum klareren Verständnis will ich zunächst den spezielleren Algorithmus formal beschreiben, und in 
\ref{s.allgemeinerAlgorithmus} bei einer Verallgemeinerung ankommen. 

Wie bereits erwähnt ähnelt das Verfahren einer Graphgrammatik. Es ist inspiriert von Joris Dormans Ansatz in 
\cite{dormansAdventures},
allerdings werden keine separaten Modelle für Missionsstruktur und Raum verwendet.
In \cite[2.]{dormansModelTransformation} redet Dormans von „mission and space“ und dass diese beiden Modelle von Level-Designern häufig als isomorphisch angenommen werde. Er argumentiert jedoch, dass sie als separate Modelle verstanden werden sollten.

Für mein Verfahren wurden keine separaten Modelle verwendet, sondern direkt ein Raum generiert, der eine Missionsstruktur implizit  beinhaltet. Für die Zwecke dieser Arbeit, wurde diese Vereinfachung als „gut genug“ befunden.

Der Generierung dieses Raumes würde ich mich gerne über das bereits bekannte Konzept der Graphgrammatik annähern. Hierfür stelle man sich einen mit Hilfe einer Graphgrammatik generierten Level-Graph vor. Dieser ist allerdings an ein dreidimensionales kartesisches Gitter gebunden. Verbindungen zwischen Knoten des Graphen sind nur zwischen direkt (nicht diagonal) benachbarten Zellen möglich, wie in Abbildung \ref{img.gridExample3D} dargestellt.

\bild{gridExample3D}{6cm}{Beispiel für einen dreidimensionalen Graph innerhalb der Restriktionen des Gitters}{Beispiel für Graph im Gitter}

Zu Jedem Knoten und den mit ihm verbundenen Kanten kann man sich einen äquivalenten Raum vorstellen. Als Verbindung zwischen den Räumen dient ein Ausgang, die äquivalent zu einer halben Kante ist (siehe Abb. \ref{img.nodeToRoom}).

\bild{nodeToRoom}{8cm}{Knoten mit Halb-Kanten und äquivalenter Raum}{Äquivalenz von Knoten und Raum}

Würde man einen solchen Raum, der genau eine Gitterzelle einnimmt für alle Ausrichtungen manuell entwerfen, etwa in einem 3D-Programm, so würden $ 2^6 = 64 $ verschiedene Raum-Vorlagen benötigt, da es sechs Ausgänge pro Raum gibt, die jeweils zwei Zustände („Tür“ oder „keine Tür“) haben können. Durch geschicktes nutzen von Rotationssymmetrie könnte man diese Zahl verringern, dennoch ist dies nicht ideal, da ein Level-Designer trotzdem eine Mindestzahl an Räumen entwerfen müsste. Außerdem will man für einen abwechslungsreichen Dungeon verschiedene „Themes“ haben, verschiedene Arten von Raumübergängen und andere Arten von Variationen. Diese Variationen kann man nicht alle in allen möglichen Eingangs/Ausgangsvariationen bereitstellen. 

\todo{Nochmal genauer darüber nachdenken, was genau diese Variation hier bedeuten, etc. (oder streichen)}

Alternativ könnte man auch die einzelnen Räume prozedural generieren. Dieser Ansatz wurde ausgetestet, aber nicht weiter verfolgt, da die Räume dadurch zu gleich aussahen. \todo{Screenshot von VoxelRooms} Einen abwechslungsreichen Raum-Generator zu entwickeln ist ein interessantes Problem, aber nicht Inhalt dieser Arbeit (mehr zu Verbesserungen siehe. \ref{s.verbesserungen}).

Die Lösung dieses Problems ist die Umformulierung des Modells. Anstatt einen Level-Graphen zu generieren, und diesen in einem zweiten Schritt in Räume zu übersetzen, werden die Räume direkt produziert. 

Dies bedeutet, dass keine Graphgrammatik im eigentlichen Sinne verwendet wird sondern stattdessen eine dreidimensionale Array-Grammatik. 

Eine Array-Grammatik verhält sich wie eine normale Formale Grammatik, unterscheidet sich jedoch dadurch, dass die linke und rechte Seite gleich große Arrays sind. 

\todo{Formale Definition Gittergrammatik, Buch suchen}

Für diesen Algorithmus wird eine Array-Grammatik verwendet, deren Produktionsregeln als linke und rechte Seite gleich große dreidimensionale Arrays haben. Ein Vorkommen der linken Seite kann durch die zugehörige rechte Seite ersetzt werden. Im Folgenden wird außerdem nicht zwischen Terminal- und Nichtterminalsymbolen unterschieden. Dies hat den Zweck, dass man die Generierung zu jeder Zeit abbrechen kann, und trotzdem eine korrekte Dungeon-Darstellung hat. Eine Unterscheidung von Terminalen und Nichtterminalen wäre aber einfach vorstellbar.

\bild{ruleExample1}{8cm}{Beispiel einer Produktionsregel\footnotemark}{Beispiel einer Produktionsregel}

\footnotetext{Arrays werden im Folgenden in 2D dargestellt, um die Grafiken übersichtlicher zu gestalten, sämtliche Illustrationen gelten im gleichen Maße für dreidimensionale Arrays}

Anders als bei eindimensionalen Grammatiken wo sich die länge des Wortes bei der Ersetzung ändern kann, ändert sich die Größe des Arrays dabei nicht, da linke und rechte Seite gleich groß sind, es wird lediglich das relevante Teilstück ersetzt.

\bild{replaceExample1}{8cm}{Anwendung der Produktionsregel A (eine von drei möglichen Positionen der Anwendung)}{Beispiel einer Regelanwendung}

Um mit dieser Array-Grammatik einen Dungeon zu erstellen müssen wir lediglich statt einem Alphabet aus Buchstaben ein Alphabet aus Raumteilen verwenden. Verwendet man ein solches Alphabet, so kann eine Graphgrammatik implizit in den Arrays enthalten sein, wie in Abbildung \ref{img.ruleExample2} illustriert. In der Praxis sind diese Raumteile 3D Modelle, die modular zusammenpassen und so kann ein ganzer Dungeon mit Regeln transformiert werden.

Hierbei ist dem Autor/der Autorin der Regeln überlassen, dass die Regeln selbst einen korrekten Dungeon produzieren. Eine Regel, die z.B. eine Verbindung ins Nichts führen lässt sollte nicht erstellt werden. 

Ein großer Vorteil dieser Methode ist, dass das Alphabet aus beliebig vielen oder wenigen Raumteilen bestehen kann. Des weiteren lässt es sich einfach für Rotationssymmetrie optimieren, indem man die Symbole zu Gruppen des gleichen Modells in verschiedenen Rotationen Zusammenfasst (\todo{Ref zu Implementierung}).

\todo{Bild aus der Implementierung nehmen}

\bild{ruleExample2}{10cm}{Alphabet $\Sigma$ mit Level-Graph-Teilen und einer Produktionsregel $R$}{Beispiel einer Produktionsregel aus Levelteilen}

\section{Allgemeiner Algorithmus}\label{s.allgemeinerAlgorithmus}

Der in Abschnitt \ref{s.allgemeinerAlgorithmus} beschriebene Algorithmus ist der in der Implementierung verwendete. 

Im Folgenden will ich diesen allerdings zu akademischen Zwecken (\todo{really?}) verallgemeinern.

$ G = (V, T, P, S) $

$ V $ ist die Symbolmenge 

$ T \subset V $ ist die Menge der Terminalsymbole

$ P \subset (V^\ast \setminus t^\ast) \times V^\ast $ ist die Menge der Produktionsregeln

$ X^\ast $ 



\todo{Mache ich später, wenn platz und bock. :P}

\chapter{Implementierung}\label{c.implementierung}

In diesem Kapitel wird die praktische Umsetzung des in \ref{s.speziellerAlgorithmus} beschriebenen Algorithmus behandelt. Die Implementierung beinhaltet viele einzelne Komponenten und Konfigurationsmöglichkeiten. Sie ist objektorientiert programmiert und läuft als Erweiterung der Unity Engine. 

\section{Ziele der Implementierung}

Wie später in \ref{s.anwendungsmöglichkeiten} beschrieben, sind viele verschiedene Anwendungsszenarien für diesen Generator denkbar. 

Ein Ziel der Implementierung ist es, dass die Software nicht einen spezifischen Generator darstellt, sondern eine Software in der sich verschiedene Generatoren der gleichen Klasse konfigurieren lassen. Es sollten möglichst große Teile des Algorithmus nicht hard-coded sondern konfigurierbar sein, indem sie als Daten vorliegen. Insbesondere gilt das für das Alphabet und die Produktionsregeln der Array-Grammatik.

Ein weiteres Ziel ist es die Features der Unity-Engine zu nutzen, Insbesondere die Entity-Component-Architektur und die Möglichkeiten den Unity-Editor mit eigener UI zu erweitern. 

\section{Wichtige Unity-Features}

Unity verwendet eine Entity-Component-Architektur. So werden an ein GameObject beliebig viele Komponenten angehängt, die verschiedene Funktionen erfüllen und miteinander interagieren können. Dies ermöglicht einen modularen Aufbau von Objekten. 
\cite[Seite: GameObjects]{unityManual}
Neue Komponenten können in Unity programmiert werden, indem man eine Klasse schreibt, die von der Klasse MonoBehaviour erbt.
\cite[Seite: CreatingAndUsingScripts]{unityManual}

Ein Beispiel hierfür kann man in Abbildung \ref{img.screenshotGenerator} sehen, nur durch das Zusammenspiel der einzelnen Komponenten ergibt sich ein Generator. Durch den modularen Aufbau verhindert man zum einen Code Duplication (Es wird z.B. der PatternView auch in anderen Objekten wiederverwendet), zum anderen ist es eine Anpassungsmöglichkeit: So lässt sich nach Belieben ein RandomReplacer oder ein RecipeReplacer verwenden um verschiedenes Verhalten zu erzeugen. 

Manche Komponenten ließen sich prinzipiell zusammenfassen, wie z.B. PatternView und PatternToPrefabs, aber wurden trotzdem getrennt implementiert, da sie unterschiedliche Aufgaben erfüllen. Dies macht auch das Ausbauen der Software in der Zukunft einfacher, da einzelne Komponenten ausgetauscht oder umgeschrieben werden können, solange die öffentlichen Funktionen und Variablen dieser Komponenten gleich bleiben.

In Abbildung \ref{img.screenshotGenerator} sieht man auch gut ein weiteres wichtiges Unity-Feature, nämlich die Darstellung von öffentlichen Variablen im Inspektor. Ist ein Field public (oder private und mit dem SerializeField-Attribute gekennzeichnet) so kann der Wert dieses Fields im Inspektor gesetzt werden; Unity unterstützt hierbei gewisse Datentypen von Haus aus, für andere kann man sogenannte Property Drawers definieren. Alle Generator-Komponenten haben public Fields, was den ganzen Generator sehr konfigurierbar macht.

Teilweise wurde außerdem das User Interface über selbst geschriebene Editor-UI ausgebaut, mehr hierzu in Abschnitt \ref{s.editorUI}.

\section{Aufbau des Generator-Objekts}

Das wichtigste GameObject der Software ist der „Generator“, dieser ist üblicherweise zusammengesetzt aus folgenden Komponenten: 
Transform (Bestandteil von jedem GameObject), PatternView, RandomReplacer oder RecipeReplacer, RuleSet, RuleEditor (optional), ReplacementEngine, PatternToPrefabs. Auf diese Komponenten wird in \ref{komponenten} genauer eingegangen.

\bild{screenshotGenerator}{12cm}{Screenshots des Inspektors eines Generator-GameObjects, man sieht alle vorhandenen Komponenten und deren Einstellungen}{Screenshot der Generator-Komponenten}

\subsection{Erklärung der Komponenten}\label{komponenten}

Es folgt eine kurze Beschreibung der einzelnen Komponenten. Es wird vor allem beschrieben, wie die Komponenten benutzt werden. \todo{Besser schreiben}

\subsubsection{PatternView}

Das PatternView repräsentiert einen dreidimensionalen Array von Labels. Diese Labels stehen üblicherweise für einen Raum und eine Rotation. Das Label „st\_2“ steht beispielsweise für eine gerades Stück („st“ für „straight“), dass um zwei mal um die y-Achse rotiert wurde. Das PatternView enthält außerdem ein Display Delta, dass bestimmt mit wie viel Abstand die Räume im Raum verteilt sind. Im Scene View\footnote{Übersicht der Unity-UI: \cite[Seite: LearningtheInterface]{unityManual}} von Unity werden die Labels als Text im dreidimensionalen Raum angezeigt (erkennbar in Abb. \ref{img.sceneView1}).

\bild{sceneView1}{16cm}{Sicht auf einen Teil eines generierten Dungeons im Unity Scene View}{Dungeon-Teil im Unity Scene View}

\subsubsection{PatternToPrefabs}

Der PatternToPrefabs Component ist ein Component, der für jedes Label im PatternView ein oder mehrere Prefabs in der Szene erstellt. Dies geschieht wenn eine Änderung im PatternView stattfindet und einmal bei Spielstart. In einem ScriptableObject\footnote{Scriptable Objects in Unity: \cite[Seite: class-ScriptableObject]{unityManual}} der Klasse PrefabDictionary wird jedem Raumnamen wie „st“ ein Prefab\footnote{Prefabs in Unity: \cite[Seite: LearningtheInterface]{unityManual}} zugeordnet. 

Es können beliebig viele Raumnamen mit Unterstrichen getrennt in einem Label vorkommen, die zugehörigen Prefabs werden dann übereinander generiert. die Kette kann mit einer Zahl beendet werden, die die Rotation in $90^\circ$-Schritten um die y-Achse angibt. Ein Beispiel hierfür ist „it\_obstacle\_0“ in Abbildung \ref{img.sceneView1}. Es wird das Kreuzung-Prefab „it“ mit einem Hindernis-Prefab „obstacle“ generiert und um 0 mal $90^\circ$ gedreht.

Der PatternToPrefabs Component verlangt ein PatternView Component mit RequireComponent\footnote{RequireComponent in Unity: \cite[Seite: RequireComponent]{unitySciptingReference}}

\subsubsection{ReplacementEngine}

Die ReplacementEngine ist verantwortlich für die im Hintergrund stattfindende Ersetzung. Sie findet im PatternView Matches für Regeln aus dem RuleSet. Erhält sie von einem anderen Component wie RandomReplacer das Signal, wendet sie eine Produktionsregel an. Dabei kann eingestellt werden ob an einer zufälligen Position, der zuerst gefundenen, oder zuletzt gefundenen Position ersetzt werden soll. Es kann außerdem ein Filter übergeben werden, sodass nur bestimmte Regeln ersetzt werden. Dieser Filter wird vom RecipeReplacer verwendet.

Der PatternToPrefabs Component verlangt die PatternView und RuleSet Components mit RequireComponent.

\subsubsection{RandomReplacer und RecipeReplacer}

Es gibt zwei verschiedene Replacer: Den RandomReplacer und den RecipeReplacer. Beide Replacer erfüllen die selbe Funktion, sie dienen als Haupt-Schnittstelle des Generators. Von ihnen kann die Generation ausgelöst werden.

Dies ist zu Testzwecken im Unity Editor möglich, indem man auf den Generate-Knopf drückt, oder zur Laufzeit indem man die Generate-Methode aufruft. Die Generate-Methode generiert einen ganzen Dungeon auf einmal. Alternativ kann man auch einmal die InitializeGeneration-Methode aufrufen und im Anschluss die GenerateStep-Methode um einzelne Ersetzungen Schritt für Schritt vorzunehmen.

Der RandomReplacer ist die simplere Variante der beiden Replacer. Er wählt immer eine zufällige Position und Regel für den nächsten Produktionsschritt. Dabei gewichtet er die einzelnen Positionen gemäß der Gewichtung der Regeln im RuleSet.

Der RecipeReplacer verwendet eine Recipe-Datei des Datentypen TextAsset\footnote{Text Assets in Unity: \cite[Seite: class-TextAsset]{unityManual}}

\subsubsection{RuleSet}

Das 

\subsubsection{RuleEditor}

\section{Wichtige Klassen}
\section{Die Pattern-Datenstruktur}
\section{Lazy updating, suchen}
\section{Editor UI}\label{s.editorUI}
\section{Regelsystem}
\section{Modelle}

\chapter{Auswertung der generierten Dungeons}

Die Qualität der generierten Dungeons soll im Folgenden bewertet werden um das Verfahren zu beurteilen. 

\section{Einschränkungen}

Es ist wohl zuerst zu sagen, dass eine objektive Bewertung der Dungeons schwer möglich ist. Dies hat mehrere Gründe.

Ein direkter Test mit Spielern und eine Auswertung über Fragebögen ist nicht möglich, da der Dungeon-Generator nicht für ein bestimmtes Spiel konzipiert würde. Mit einem Test-Spiel wäre dies jedoch ein guter Ansatz, wenn auch etwas aufwändig und nicht im Rahmen dieser Arbeit machbar.

Zweitens ist die Qualität von Dungeons prinzipiell subjektiv, was eine rein statistische Auswertung quasi unmöglich, bzw. wenig aussagekräftig macht.

Drittens gibt es keine akzeptierten Standards für die Auswertung von Dungeons, was einen Vergleich mit anderen Verfahren schwer macht. Eine Vergleich von verschiedenen Verfahren wäre natürlich interessant, aber auch nicht Inhalt dieser Arbeit \todo{Verweis auf Ausblick o.ä.}

\section{qualitative Auswertung}

\section{quantitative Auswertung}

% evtl. Laufzeit betrachten. Evtl. statistische Metriken von Ähnlichkeit herausfinden

\section{Mögliche Verbesserungen}\label{s.verbesserungen}

\section{Mögliche Anwendungsszenarien}\label{s.anwendungsmöglichkeiten}

\subsection{Roguelike, Rogue-lite}

\subsection{Minigolf}

\subsection{Text adventure}


\chapter{Zusammenfassung}\label{c.zusammenfassung}


% Anhang
\chapter{Verwendete Materialien}
\LaTeX -Vorlage von Martin Bretschneider 
\newblock \url{https://www.bretschneidernet.de/tips/thesislatex-vorlagen.html#vorlagen}, 2006. --
\newblock letzter Zugriff: 08.01.2019

\bibliographystyle{alphadin_martin}
\bibliography{bibliographie}
\bibliographystyleimg{alphadin_martin}
\bibliographyimg{bildquellenverzeichnis}

\chapter{Erklärung}

Hiermit versichere ich, dass ich die vorliegende Arbeit selbstständig verfasst und keine anderen als die angegebenen Quellen und Hilfsmittel benutzt habe, dass alle Stellen der Arbeit, die wörtlich oder sinngemäß aus anderen Quellen übernommen wurden, als solche kenntlich gemacht und dass die Arbeit in gleicher oder ähnlicher Form noch keiner Prüfungsbehörde vorgelegt wurde.

\vspace{3cm}
Ort, Datum \hspace{5cm} Unterschrift \\

\end{document}