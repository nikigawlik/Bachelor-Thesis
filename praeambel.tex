\documentclass[fontsize=12pt, paper=a4, headinclude, twoside=false, parskip=half+, pagesize=auto, numbers=noenddot, open=right, toc=listof, toc=bibliography]{scrreprt}
% PDF-Kompression
\pdfminorversion=5
\pdfobjcompresslevel=1
% Allgemeines
\usepackage[automark]{scrpage2} % Kopf- und Fußzeilen
\usepackage{amsmath,marvosym} % Mathesachen
\usepackage[T1]{fontenc} % Ligaturen, richtige Umlaute im PDF
\usepackage[utf8]{inputenc}% UTF8-Kodierung für Umlaute usw.
% Schriften
\usepackage{mathpazo} % Palatino für Mathemodus
%\usepackage{mathpazo,tgpagella} % auch sehr schöne Schriften
\usepackage{setspace} % Zeilenabstand
\onehalfspacing % 1,5 Zeilen
% Schriften-Größen
\setkomafont{chapter}{\Huge\rmfamily} % Überschrift der Ebene
\setkomafont{section}{\Large\rmfamily}
\setkomafont{subsection}{\large\rmfamily}
\setkomafont{subsubsection}{\large\rmfamily}
\setkomafont{chapterentry}{\large\rmfamily} % Überschrift der Ebene in Inhaltsverzeichnis
\setkomafont{descriptionlabel}{\bfseries\rmfamily} % für description-Umgebungen
\setkomafont{captionlabel}{\small\bfseries}
\setkomafont{caption}{\small}
% Sprache: Deutsch
\usepackage[ngerman]{babel} % Silbentrennung
% PDF
\usepackage[ngerman,pdfauthor={Nils Gawlik},  pdfauthor={Nils Gawlik}, pdftitle={TODO}, breaklinks=true,baseurl={}]{hyperref}
\hypersetup{
    colorlinks=true,
    linkcolor=blue,
    filecolor=magenta,      
    urlcolor=cyan,
}
\urlstyle{same}

\usepackage[final]{microtype} % mikrotypographische Optimierungen
\usepackage{url} % ermögliche Links (URLs)
\usepackage{pdflscape} % einzelne Seiten drehen können
% Tabellen
\usepackage{multirow} % Tabellen-Zellen über mehrere Zeilen
\usepackage{multicol} % mehrere Spalten auf eine Seite
\usepackage{tabularx} % für Tabellen mit vorgegeben Größen
\usepackage{longtable} % Tabellen über mehrere Seiten
\usepackage{array}
%  Bibliographie
\usepackage{bibgerm} % Umlaute in BibTeX
% Bilder
\usepackage{graphicx} % Bilder
\usepackage{color} % Farben
\graphicspath{{images/}} % Lege den Standardpfad mit Bilder fest
\DeclareGraphicsExtensions{.pdf,.png,.jpg} % bevorzuge pdf-Dateien vor den anderen
\usepackage{subcaption}  % mehrere Abbildungen nebeneinander/übereinander
\usepackage[all]{hypcap} % Beim Klicken auf Links zum Bild und nicht zu Caption gehen
% Bildunterschrift
\setcapindent{0em} % kein Einrücken der Caption von Figures und Tabellen
\setcapwidth{0.9\textwidth} % Breite der Caption nur 90% der Textbreite, damit sie sich vom restlichen Text abhebt
\setlength{\abovecaptionskip}{0.2cm} % Abstand der zwischen Bild- und Bildunterschrift
% Quellcode
% für Formatierung in Quelltexten, hier im Anhang
\usepackage{listings}
\definecolor{grau}{gray}{0.25}
\lstset{
	extendedchars=true,
	basicstyle=\tiny\ttfamily,
	%basicstyle=\footnotesize\ttfamily,
	tabsize=2,
	keywordstyle=\textbf,
	commentstyle=\color{grau},
	stringstyle=\textit,
	numbers=left,
	numberstyle=\tiny,
	% für schönen Zeilenumbruch
	breakautoindent  = true,
	breakindent      = 2em,
	breaklines       = true,
	postbreak        = ,
	prebreak         = \raisebox{-.8ex}[0ex][0ex]{\Righttorque},
}
% linksbündige Fußnoten
\deffootnote{1.5em}{1em}{\makebox[1.5em][l]{\thefootnotemark}}

\typearea{14} % typearea berechnet einen sinnvollen Satzspiegel (das heißt die Seitenränder usw.) siehe auch http://www.ctan.org/pkg/typearea. Diese Berechnung befindet sich am Schluss, damit die Einstellungen von oben berücksichtigt werden

\usepackage{scrhack} % Vermeidung einer Warnung


% Eigene Befehle %%%%%%%%%%%%%%%%%%%%%%%%%%%%%%%%%%%%%%%%%%%%%%%%%5
% Matrix
\newcommand{\mat}[1]{
      {\textbf{#1}}
}
\newcommand{\todo}[1]{
      {\colorbox{red}{ TODO: #1 }}
}
\newcommand{\todotext}[1]{
      {\color{red} TODO: #1} \normalfont
}
\newcommand{\info}[1]{
      {\colorbox{blue}{ (INFO: #1)}}
}
% Hinweis auf Programme in Datei
\newcommand{\datei}[1]{
      {\ttfamily{#1}}
}
\newcommand{\code}[1]{
      {\ttfamily{#1}}
}
% Bild mit definierter Breite einfügen
\newcommand{\bild}[4]{
  \begin{figure}[!hbt]
    \centering
      \vspace{1ex}
      \includegraphics[width=#2]{images/#1}
      \caption[#4]{\label{img.#1} #3}
    \vspace{1ex}
  \end{figure}
}
% Bild mit eigener Breite
\newcommand{\bilda}[3]{
  \begin{figure}[!hbt]
    \centering
      \vspace{1ex}
      \includegraphics{images/#1}
      \caption[#3]{\label{img.#1} #2}
      \vspace{1ex}
  \end{figure}
}

% Bild todo
\newcommand{\bildt}[2]{
  \begin{figure}[!hbt]
    \begin{center}
      \vspace{2ex}
	      \includegraphics[width=6cm]{images/todobild}
      %\caption{\label{#1} \color{red}{ TODO: #2}}
      \caption{\label{#1} \todotext{#2}}
      %{\caption{\label{#1} {\todo{#2}}}}
      \vspace{2ex}
    \end{center}
  \end{figure}
}
